\documentclass{article}
\usepackage[T1]{fontenc}
\usepackage{lato}
\usepackage{graphicx}
\RequirePackage{polski}
\graphicspath{ {./images/} }

\title{Laboratorium przetwarzanie obrazów}
\date{}
\author{Radosław Piotrowicz}


\begin{document}
\maketitle

\begin{center}
    \large{ Informacje potrzebne do implementacji }
\end{center}

\newpage


\section{Gdzie znajdują się interesujące nas rzeczy.}
Wszystkie algorytmy należy implementować w \textbf{Algorithms.cpp}.
\newline
W \textbf{Algorithms.hpp} znajdują się wszystkie deklaracje funkcji.
\newline
W \textbf{Algorithms.hpp} znajduje się również definicja struktury z parametrami do algorytmów (niektóre nie wymagają parametrów).

\begin{figure}[htbp!]
    \centering
    \includegraphics[width=0.8\textwidth]{declarations.png}
    \caption{Deklaracje funkcji}
\end{figure}

\begin{figure}[htbp!]
    \centering
    \includegraphics[width=0.8\textwidth]{params.png}
    \caption{Struktura danych z parametrami}
\end{figure}

\newpage

\section{Co jest potrzebne przy implementacji algorytmu}

Algorytmy wykonują się w oddzielnym wątku, w związku z czym w funkcji muszą znaleźć się dwie rzeczy.

\subsection{Lokalna kopia obrazu i parametrów}

\begin{figure}[htbp!]
    \centering
    \includegraphics[width=0.8\textwidth]{CopyToLocal2.png}
    \caption{Kopiowanie potrzebnych zasobów}
\end{figure}

\subsection{Skopiowanie wyników do zasobów współdzielonych oraz informacja, że wątek się zakończył}
\begin{figure}[htbp!]
    \centering
    \includegraphics[width=0.8\textwidth]{SaveAtTheEnd.png}
    \caption{Zatrzymanie i skopiowanie wyników}
\end{figure}

Można też opcjonalnie dodać odświeżanie i możliwość przerwania wykonywania.
\newline
\textbf{ Natomiast nie jest to wymagane.}
\newline
Automatyczne odświeżanie można włączyć w zakładce Ustawienia.
Tam również można ustawić co ile sekund ma się odświeżać

\newpage

\subsection{Jak dodać przerwanie i odświeżanie do funkcji}

\begin{figure}[htbp!]
    \centering
    \includegraphics[width=0.8\textwidth]{RefreshAndCancel.png}
    \caption{Odświeżanie i anulowanie}
\end{figure}

W niektórych algorytmach efekt jest lepszy, jeżeli ustawimy ręczne odświeżanie.
Wtedy można to zrobić w ten sposób.
\begin{figure}[htbp!]
    \centering
    \includegraphics[width=0.8\textwidth]{CancelAndManualRefresh.png}
    \caption{Zatrzymanie i ręczne kopiowanie}
\end{figure}

\newpage

\section{Opis klas i przydatnych metod}

\subsection{Klasa Image}
Klasa Image jest kontenerem na obrazy, które będziemy przetwarzać.
\newline
Najbardziej interesuje nas w niej SDLSurface, czyli struktura z informacjami o pikselach w obrazie.
\newline
Zawiera takie przydatne metody jak:
\begin{itemize}
    \item  \textbf{Image} operator=(const Image other) - kopiuje całą zawartość innego obiektu
    \item \textbf{int} GetWidth() - zwraca szerokość obrazu
    \item \textbf{int} GetHeight() - zwraca wysokość obrazu
    \item \textbf{int} GetPixelCount() - zwraca ilość pikseli
    \item \textbf{float*} GetLightValues() - zwraca tablice float o długości 256 z wartościami jasności (są też wersje dla poszczególnych składowych)
    \item \textbf{float*} GetDistributor() - zwraca tablice float o długości 256 z wartościami dystrybuanty (są też wersje dla poszczególnych składowych)
    \item \textbf{void} CopyBrightnessHistogram(float *dst) - kopiuje histogram do podanej tablicy
    \item \textbf{void} CopyNormalisedBrightnessHistogram(float *dst) - kopiuje histogram (unormowany do wartości 0.0 - 1.0) do podanej tablicy
    \item \textbf{bool} NoSurface() - czy obraz zawiera poprawnie zainicjowane SDLSurface (jeśli nie to źle)
    \item \textbf{void} SetBlankSurface(int width, int height) - ustaw obraz na biały prostokąt o wymiarach
    \item \textbf{void} RefreshPixelValuesArrays() - odśwież tablice z histogramami i dystrybuantami (po poprawnym zakończeniu wątku z algorytmem wątek główny automatycznie odświeży)
    \item \textbf{Pixel} GetPixel(int x, int y) - zwraca Strukturę Pixel z informacjami o pikselu w danym punkcie
    \item \textbf{void} SetPixel(int x, int y, Pixel pix) - ustawia pixel w danym punkcie na podany
    \item \textbf{void} SetPixelWhite(int x, int y) - ustawia pixel w danym punkcie na kolor biały
    \item \textbf{void} SetPixelBlack(int x, int y) - ustawia pixel w danym punkcie na kolor czarny
    \item \textbf{void} CopyOnlySurfaceAndSize(Image other) - kopiuje tylko surface i wymiary (użyteczne w algorytmach, w których wartość jednego piksela zależy od wartości jego sąsiadów)
\end{itemize}

\subsection{Struktura Pixel}
Ta struktura zawiera wartości R, G, B oraz jasność piksela.
\newline
\textbf{Należy pamiętać, że gdy modyfikujemy piksel, należy ustawić wszystkie 3 składowe.}
\newline
\textbf{WARTOŚĆ JASNOŚCI JEST TYLKO DO ODCZYTU. METODA SetPixel NIE BIERZE JEJ POD UWAGĘ.}

\subsection{Klasa Mutex}
Semafor do synchronizacji wątków
\begin{itemize}
    \item  \textbf{void} Lock() - opuść semafor
    \item \textbf{void} Unlock() - podnieś semafor
    \item \textbf{bool} IsThreadRunning() - czy wątek przetwarzający ma się dalej wykonywać (używane przy przerwaniu wykonywania wątku)
    \item \textbf{void} ThreadStopped() - ustawia, że wątek przetwarzający zakończył wykonywanie
\end{itemize}



\end{document}
