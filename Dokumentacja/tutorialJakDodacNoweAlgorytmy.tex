\documentclass{article}
\usepackage[T1]{fontenc}
\usepackage{lato}
\usepackage{graphicx}
\usepackage{hyperref}
\RequirePackage{polski}
\graphicspath{ {./images/tutorial} }

\title{Laboratorium przetwarzanie obrazów}
\date{}


\begin{document}
\maketitle

\begin{center}
    \large{ Jak rozszerzyć program o nowe algorytmy }
\end{center}

\newpage


\section{Gdzie znajdują się interesujące nas rzeczy.}
Jeżeli chcemy dodać nowy algorytm należy wprowadzić zmiany w plikach:
\begin{itemize}
    \item \textbf{App.hpp} - dodanie do enum \textbf{AlgSelected}, nowgo algorytmu. Oraz opcjonalnie deklaracje metod pomocniczych.
    \item \textbf{App.cpp} - dokonanie zmian w metodach \textbf{DrawAlgMenuElements, DrawParametersPopup, LaunchAlgorithms, ResetParameters(opcjonalne)}
    \item \textbf{Algorithms.hpp} - dodanie deklaracji nowej funkcji analogicznie do już tam zawartych. Dodanie nowych parametrów do \textbf{ParametersStruct} jeżeli są wymagane. Dodanie kolejego enum, jeżeli będzie potrzebne.
    \item \textbf{Algorithms.cpp} - implementacja nowego algorytmu.
\end{itemize}

\section{App.hpp}
Pierwszym krokiem w dodaniu nowego algorytmu jest dodanie go do enum \textbf{AlgSelected}.

\begin{figure}[htbp!]
    \centering
    \includegraphics[width=0.8\textwidth]{AlgSelectegEnum.png}
    \caption{Enum AlgSelected}
\end{figure}

\newpage

\section{App.cpp}
Następnie należy wprowadzić zmiany w metodach.
\subsection{DrawAlgMenuElements}

\begin{figure}[htbp!]
    \centering
    \includegraphics[width=0.8\textwidth]{DrawAlgMenu.png}
    \caption{Metoda DrawAlgMenu}
\end{figure}

Tutaj należy dodać na końcu kolejny \textbf{if} analogicznie do już zawartych.
\newline
Każdy \textbf{if} jest po to aby wykryć kliknięcie w dany element w menu wyboru algorytmów.

\newpage

\begin{figure}[htbp!]
    \centering
    \includegraphics[width=0.8\textwidth]{AlgMenu.png}
    \caption{Menu wyboru algorytmów}
\end{figure}

\newpage

\subsection{DrawParametersPopup}
Tutaj należy do \textbf{switch} dodać nowy przypadek. W zależności od algorytmu ilość i typ parametrów mogą się różnić.

\begin{figure}[htbp!]
    \centering
    \includegraphics[width=0.8\textwidth]{DrawParametersPopup.png}
    \caption{Metoda DrawParametersPopup}
\end{figure}

Jeżeli algorytm nie ma parametrów należy dodać: \textbf{ImGui::Text("Brak parametrów dla tego algorytmu.")}
\newline
Tak będzie wyglądało okienko.
\begin{figure}[htbp!]
    \centering
    \includegraphics[width=0.8\textwidth]{NoPrams.png}
    \caption{Okienko bez parametrów.}
\end{figure}

\newpage

W innym przypadku należy dostosować okienko. Oto kilka przykładów:
\subsubsection{Pojedyńczy parametr.}
Na Rysunku 4 było widać, że dla niektórych algorytmów pojawiała się obsługa tylko jednego parametru na przykład:
\newline
\textbf{ImGui::SliderInt("O ile rozjaśnić/przyciemnić?", \&params.value, -255, 255)}
\newline
Najpierw jest opis slidera, potem wartość do której zapisany będzie wynik, następnie ograniczenia dolne i górne.
\newline
A tak będzie wyglądało okienko.
\begin{figure}[htbp!]
    \centering
    \includegraphics[width=0.8\textwidth]{SimpleParam.png}
    \caption{Pojedyńczy parametr}
\end{figure}

\subsubsection{Kilka parametrów}
Czasami może być potrzebne kilka prametrów do algorytmu. Takich jak rodzaj metody lub liczba progów.
Wtedy można do zrobić w ten sposób.

\newpage

\begin{figure}[htbp!]
    \centering
    \includegraphics[width=0.8\textwidth]{Binarization.png}
    \caption{Kilka parametrów}
\end{figure}

Na początku metody pomocniczej znajduje się pierwszy wybór (metody), obsługiwany jest przez funkcje \textbf{ImGui::RadioButton()}.
Funkcja ta przyjmuje jako parametry:
\begin{itemize}
    \item  \textbf{const char *label} - nazwa opcji
    \item  \textbf{int *v} - wskaźnik do zmiennej z obecnie wybranym stanem
    \item  \textbf{int v\_button} - numer stanu przypisany do opcji (najlepiej zdefiniowany w enum)
\end{itemize}

\textbf{ImGui::SameLine()} - informuje żeby umieścić elementy w tej samej lini.
\newline
\textbf{ImGui::Separator()} - tworzy linię separującą elementy.
\newline
Następnie znajduje się kolejny wybór (ilości progów), zrobiony analogicznie do wyboru metody. Jedak pojawi się on tylko wtedy gdy ustawiamy progi ręcznie.
Również jeżeli ustawiamy progi ręcznie pojawią się dwa slidery do wyboru wartości progu.
\newpage
Okienko będzie wyglądało tak:

\begin{figure}[htbp!]
    \centering
    \includegraphics[width=0.8\textwidth]{BinarizationP.png}
    \caption{Okienko kilku parametrów}
\end{figure}

\subsubsection{Patametry tabelkowe}
Przy niektórych algorytmach parametrem jest macierz (maska albo element strukturalny).

\begin{figure}[htbp!]
    \centering
    \includegraphics[width=0.8\textwidth]{InputArrayExample.png}
    \caption{Tabelka w kodzie}
\end{figure}

Aby rozpocząć tabelkę należy użyć funkcji ImGui::BeginTable(), i musi się ona znajdować w instrukcji warunkowej.
Funkcja ta przyjmuje następujące parametry:
\begin{itemize}
    \item \textbf{const char *str\_id} - nazwa tablicy
    \item \textbf{int columns} - ilość kolumn
    \item \textbf{ImGuiTableFlags flags} - flagi dotyczące wyglądu tablicy
    \item \textbf{const ImVec2 \&outer\_size} - rozmiar tablicy
    \item \textbf{float inner\_width} - wewnętrzna szerokość komórek
\end{itemize}
W naszym przypadku interesować będą nas pierwsze 3-4 parametry.

Następnie należy w pętli podwójnej utworzyć wszystkie elementy.

W zewnętrznej pętli (dla wierszy) używamy ImGui::TableNextRow() aby przejść do kolejnego wiersza.
Następnie w wewnętrzej pętli (dla kolumn) ustawiamy kursor na kolumnę przy użyciu ImGui::TableSetColumnIndex(col), i dodajemy element.
\newline
Aby dodać nowy element moża użyć różnych elementów na przykład:
\begin{itemize}
    \item ImGui::Text() - do wyświetlenia elementu bez możliwości wprowadzenia.
    \item ImGui::InputInt() - do wprowadzenia liczby całkowitej z powiązaniem do elementu o odpowiednim indeksie.
    \item ImGui::Checkbox() - do oznaczenia elementu którego wartości mogą przyjąć true/false.
\end{itemize}

Tak będą wyglądać tablice w kolejności takiej jak na liście:

\newpage

\begin{figure}[htbp!]
    \centering
    \includegraphics[width=0.8\textwidth]{FilterP.png}
    \caption{Tablica z samym wyświetlaniem zawartości}
\end{figure}

\begin{figure}[htbp!]
    \centering
    \includegraphics[width=0.8\textwidth]{FilterPC.png}
    \caption{Tablica z wprowadzaniem liczb całkowitych}
\end{figure}

\newpage

\begin{figure}[htbp!]
    \centering
    \includegraphics[width=0.8\textwidth]{Element.png}
    \caption{Tablica z checkboxami}
\end{figure}

Na koniec tablice należy zakończyć używając ImGui::EndTable().

\textbf{Dodatkowe uwagi}
\begin{itemize}
    \item Jeżeli tablica ma służyć do wprowadzania danych należy użyć \textbf{ImGui::PushID(nr wiersza)} - przed ImGui::TableNextRow() i \textbf{ImGui::PopID()} - po wewnętrznej pętli.
    \item Jeżeli ustawiamy ID należy najpierw użyć \textbf{std::string s = "\#\#" + std::to\_string(col)} aby ustalić to ID, a następnie przkazać je do funkcji od elemenmtu na przykład \textbf{ImGui::Checkbox(s.c\_str(), \&a3x3[row][col])}.
    \item Jeżeli nie potrzebujemy opisów kolumn, można użyć \textbf{ ImGui::TableSetupColumn("\#\#")} dla każdej kolumny.
    \item Jeżeli chcemu ustawić szerokość elementu użyjemy \textbf{ImGui::TableSetupColumn("\#\#", ImGuiTableColumnFlags\_WidthFixed, ARRAY\_ITEM\_WIDTH)} \textbf{Tablica powinna mieć podany stały rozmiar w ImGui::BeginTable()}.
    \item Jeżeli tablica nie ma ustawionego rozmiaru na początku można określić szerokość elementu możemy użyć \textbf{ImGui::PushItemWidth(ARRAY\_INPUT\_WIDTH)}.
    \item Metoda \textbf{DrawInputArray()}, może być wykorzystywana jeżeli chcemy utworzyć kolejne tablice z checboxami.
    \item \textbf{Jeżeli fragment kodu od obsługi parametrów jest zbyt długi można go przenieść do metody pomocniczej.}
\end{itemize}



\subsection{LaunchAlgorithms}
W tej metodzie należy dodać kolejny przypadek analogicznie do już instniejących.
\newline
\textbf{Można też wywołać funkcje w wątku głównym natomiast wtedy nie będzie możliwości przerwania, a interfejs nie będzie odpowiadał. Należy też użyć Mutex::GetInstance().ThreadStopped(), aby popup pokazujący się w trakcie wykonywania nie wyświetlał się. }

\newpage

\begin{figure}[htbp!]
    \centering
    \includegraphics[width=0.8\textwidth]{LaunchAlgorithms.png}
    \caption{Metoda LaunchAlgorithms}
\end{figure}

\subsection{ResetParameters}
W tej metodzie należy ustawić parametry na wartości domyślne (ustalane w strukturze z parametrami), jeżeli algorytm posiada parametry.

\newpage

\begin{figure}[htbp!]
    \centering
    \includegraphics[width=0.8\textwidth]{ResetParameters.png}
    \caption{Metoda ResetParameters}
\end{figure}

\section{Algorithms.hpp}
Tutaj należy zadeklarować nową funkcję dla kolejnego algorytmu. Tutaj również znajduje się definicja struktury z parametrami, więc jeżeli trzeba dodać nowe to również tutaj się znajdują. Można też dodać pomocniczy enum jeżeli będzie potrzebny.
\newline
\textbf{Najlepiej zacząć dodanie nowego algorytmu od tego pliku.}

\newpage

\begin{figure}[htbp!]
    \centering
    \includegraphics[width=0.8\textwidth]{AlgorithmsFunctionsDec.png}
    \caption{Deklaracje funkcji}
\end{figure}


\begin{figure}[htbp!]
    \centering
    \includegraphics[width=0.8\textwidth]{HelperEnums.png}
    \caption{Pomocnicze enumy}
\end{figure}

\newpage

\begin{figure}[htbp!]
    \centering
    \includegraphics[width=0.8\textwidth]{ParametersStruct.png}
    \caption{Struktura z parametrami}
\end{figure}

\newpage

\section{Algorithms.cpp}
Tutaj należy zdefiniwać nowy algorytm. Należy pamiętać, że domyślnie powinien się wykonywać w oddzielnym wątku. Natomiast moża funkcje wywołać w wątku głównym.
\newline
Należy zwrócić uwagę na:

\subsection{Lokalna kopia obrazu i parametrów}

\begin{figure}[htbp!]
    \centering
    \includegraphics[width=0.8\textwidth]{CopyToLocal2.png}
    \caption{Kopiowanie potrzebnych zasobów}
\end{figure}

\subsection{Skopiowanie wyników do zasobów współdzielonych oraz informacja że wątek się zakończył}
\begin{figure}[htbp!]
    \centering
    \includegraphics[width=0.8\textwidth]{SaveAtTheEnd.png}
    \caption{Zatrzymanie, i skopiowanie wyników}
\end{figure}

Moża też opcjonalnie dodać odświeżanie i możliwość przerwania wykonywania.
\newline
\textbf{ Natomiast nie jest to wymagane.}
\newline
Automatyczne odświerzanie można włączyć w zkładcę Ustawienia.
Tam również można ustawić co ile sekund ma się odświerzać

\newpage

\subsection{Jak dodać przerwanie i odświerzanie do funkcji}

\begin{figure}[htbp!]
    \centering
    \includegraphics[width=0.8\textwidth]{RefreshAndCancel.png}
    \caption{Odświerzanie i anulowanie}
\end{figure}

W niektórych algorytmach efekt jest lepszy jeżeli ustawimy ręczne odświerzanie.
Wtedy można to zrobić w ten sposób.
\begin{figure}[htbp!]
    \centering
    \includegraphics[width=0.8\textwidth]{CancelAndManualRefresh.png}
    \caption{Zatrzymanie, i ręczne kopiowanie}
\end{figure}

\section{ImGui}
W tej sekcji opisze przydatne informacje dotyczące użytej biblioteki graficznej.
\subsection{ImGuiDemo}
Do programu dołączone jest ImGuiDemo, które zawiera przykłady możliwości biblioteki graficznej.
\newline
Znajduje się ono w sekcji \textbf{Pomoc->Pokaż ImGuiDemo}
\newline
Okno wygląda tak:
\begin{figure}[htbp!]
    \centering
    \includegraphics[width=0.8\textwidth]{ImGuiDemo.png}
    \caption{Okno ImGuiDemo}
\end{figure}

\newpage

Przykłady w kodzie znajdują się w pliku \textbf{imgui\_demo.cpp.} Plik ten znaduje się w folderze ImGui.
\newline
W tym pliku powinno znajdować się wszystko co może być potrzebne przy zapozanawaniu się z biblioteką. Ponadto twórcy ImGui zachęcają do zachowania tego pliku w projektach w celu posiadania przykładów implemetacji funkcjonalności.

\begin{figure}[htbp!]
    \centering
    \includegraphics[width=0.8\textwidth]{ImGuiDir.png}
    \caption{Folder ImGui}
\end{figure}

\subsection{Link do biblioteki}
Na końcu zostawiam link do Githuba, z biblioteką. Znajdują się tam wszystkie potrzebne informacje.
\newline
\textbf{Github:} \href{https://github.com/ocornut/imgui}{https://github.com/ocornut/imgui}

\end{document}
