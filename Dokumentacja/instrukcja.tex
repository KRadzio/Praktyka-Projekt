\documentclass{article}
\usepackage[T1]{fontenc}
\usepackage{lato}
\usepackage{graphicx}
\RequirePackage{polski}
\graphicspath{ {./images/} }

\title{Laboratorium przetwarzanie obrazów}
\date{}


\begin{document}
\maketitle

\begin{center}
  \large{ Instrukcja konfiguracji środowiska}
\end{center}

\newpage

\section{Windows}

\subsection{}
Krokiem numer jeden jest uruchomienie Microsoft Visual Studio 22.
I otworzenie pliku projektu.

\subsection{}
Następnie należy otworzyć \textbf{Właściwości projektu}.
Najpierw w zakładce \textbf{General} należy sprawdzić, czy standard jest ustawiony na C++17.
\begin{figure}[htbp!]
    \centering
    \includegraphics[width=0.8\textwidth]{Standard.png}
    \caption{Wybór standardu}
\end{figure}

\newpage

\subsection{}
Kolejnym krokiem jest sprawdzenie, czy zakładki:
\begin{itemize}
  \item C/C++/General/Additional Include Directories
  \item Linker/General/Additional Library Directories
  \item Linker/Input/Additional Dependecies
\end{itemize}

wyglądają tak jak na obrazkach


\begin{figure}[htbp!]
  \centering
  \includegraphics[width=0.8\textwidth]{IncludePath.png}
  \caption{C/C++/General/Additional Include Directories}
\end{figure}

\newpage

\begin{figure}[htbp!]
  \centering
  \includegraphics[width=0.8\textwidth]{LibPath.png}
  \caption{Linker/General/Additional Library Directories}
\end{figure}


\begin{figure}[htbp!]
  \centering
  \includegraphics[width=0.8\textwidth]{Deps.png}
  \caption{Linker/Input/Additional Dependecies}
\end{figure}

\subsection{}
Teraz należy sprawdzić, czy możemy edytować zmienne środowiskowe. Należy wpisać w wyszukiwarkę systemową \textbf{Edytuj zmienne środowiskowe.}
\newline
Jeżeli otworzyły się odpowiednie ustawienia, wyświetli się takie okienko.
Należy wybrać opcje \textbf{Zmienne środowiskowe}

\begin{figure}[htbp!]
  \centering
  \includegraphics[width=0.8\textwidth]{MenuWizard.png}
  \caption{Okienko}
\end{figure}

Następnie należy znaleźć zmienną Path w sekcji zmiennych systemowych.
\newline
I kliknąć \textbf{Edytuj}

\newpage

\begin{figure}[htbp!]
  \centering
  \includegraphics[width=0.8\textwidth]{Vars.png}
  \caption{Okienko zmiennych}
\end{figure}

Otworzy się takie okienko.

\newpage

\begin{figure}[htbp!]
  \centering
  \includegraphics[width=0.8\textwidth]{Paths.png}
  \caption{Zmienna Path}
\end{figure}

Należy wybrać opcje \textbf{Nowa} i wpisać ścieżki do folderów zawierających pliki .dll.
\newline
\textbf{SĄ TO FOLDERY PODANE W Linker/General/Additional Library Directories}

\subsection{}
Jeżeli nie można edytować zmiennych środowiskowych (a w laboratorium nie można) to należy otworzyć \textbf{Wiersz poleceń} i wpisać \textbf{path}
\newline
Wyświetli się zmienna Path.

\newpage

\begin{figure}[htbp!]
  \centering
  \includegraphics[width=0.8\textwidth]{Console.png}
  \caption{Zmienna Path w wierszu poleceń}
\end{figure}

Należy wybrać jeden z folderów i wkleić tam pliki .dll.
\newline
\textbf{SĄ TO FOLDERY PODANE W Linker/General/Additional Library Directories}

\newpage
\section{Linux}

\subsection{}
Najpierw należy sprawdzić, czy zainstalowane są pakiety
\begin{itemize}
  \item g++
  \item make
  \item SDL2
  \item SDL2\_image-devel
  \item libSDL2-2\_0-0
\end{itemize}

Jeśli ich brakuje, to należy je zainstalować.
\newline
Jeżeli pakiety nazywają się inaczej, a dla różnych dystrybucji może tak być, należy wpisać w terminalu:
\newline
\textbf{sudo <nazwa menadżera pakietów> search sdl2}
\newline
I zainstalować odpowiadające wymienionym powyżej pakiety (mogą mieć końcówkę -dev).
\newline
\textbf{Jeżeli dalej brakuje jakichś pakietów albo przy uruchomieniu pojawi się błąd, że brakuje biblioteki, to należy ją pobrać.}

\subsection{}
Następnie należy uruchomić terminal w folderze z wersją programu na Linuxa i wpisać \textbf{make} w terminalu dla wersji relase.
Lub \textbf{make debug} dla wersji debug.
\newline
Aby uruchomić, należy wpisać \textbf{./bin/relase/prog} dla wersji relase.
\newline
Albo \textbf{./bin/debug/prog} dla wersji debug.

\end{document}
